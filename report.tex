\documentclass[a4paper,12pt]{article}
\usepackage[UTF8]{ctex}     % 中文支持
\usepackage{amsmath, amssymb, amsfonts} % 数学公式
\usepackage{graphicx}       % 插入图片 (draft模式不加载图片,防止卡顿)
\usepackage{float}          % 图片浮动控制
\usepackage{geometry}       % 页面边距
\usepackage{listings}       % 代码块
\usepackage{xcolor}         % 颜色支持
\usepackage{booktabs}       % 三线表
\usepackage{hyperref}       % 超链接引用

% 页面设置
\geometry{left=2.5cm,right=2.5cm,top=2.5cm,bottom=2.5cm}

% 代码块样式设置
\definecolor{codegreen}{rgb}{0,0.6,0}
\definecolor{codegray}{rgb}{0.5,0.5,0.5}
\definecolor{codepurple}{rgb}{0.58,0,0.82}
\definecolor{backcolour}{rgb}{0.95,0.95,0.92}

\lstdefinestyle{mystyle}{
    backgroundcolor=\color{backcolour},   
    commentstyle=\color{codegreen},
    keywordstyle=\color{magenta},
    numberstyle=\tiny\color{codegray},
    stringstyle=\color{codepurple},
    basicstyle=\ttfamily\small,
    breakatwhitespace=false,         
    breaklines=true,
    postbreak=\mbox{\textcolor{red}{$\hookrightarrow$}\space},
    captionpos=b,                    
    keepspaces=true,                 
    numbers=left,                    
    numbersep=5pt,                  
    showspaces=false,                
    showstringspaces=false,
    showtabs=false,                  
    tabsize=4,
    language=Python,
    columns=flexible
}
\lstset{style=mystyle}


\begin{document}

\begin{titlepage}
    \centering
    \vspace*{3cm}
    
    {\LARGE\textbf{多媒体信息处理作业}\par}
    
    \vspace{2cm}
    
    {\large 姓名:张宗桪\par}
    {\large 专业:人工智能\par}
    {\large 学号:2023K8009991013\par}
    
    \vfill
    
    
\end{titlepage}

\tableofcontents
\newpage

\section{实验背景与任务}
\subsection{CIFAR-10 数据集}
CIFAR-10 是计算机视觉领域经典的图像分类数据集,包含 10 个类别,共 60000 张 $32 \times 32$ 彩色图像。其中训练集 50000 张,测试集 10000 张。

\subsection{ResNet 网络架构}
ResNet 通过引入“残差学习”解决了深层网络中的梯度消失和退化问题。本实验采用 ResNet-18 作为基础架构。
\begin{figure}[H]
    \centering
    \includegraphics[width=0.8\textwidth,angle=-90]{figures/resnet18.png}
    \caption{ResNet-18 结构示意图}
    \label{fig:resnet_block}
\end{figure}
\begin{lstlisting}[caption=ResNet 核心代码, label=code:ResNet-18]
class BasicBlock(nn.Module):
    expansion = 1

    def __init__(self, in_planes, planes, stride=1, activation_class=nn.ReLU):
        super(BasicBlock, self).__init__()
        self.conv1 = nn.Conv2d(in_planes, planes, kernel_size=3, stride=stride, padding=1, bias=False)
        self.bn1 = nn.BatchNorm2d(planes)
        self.act1 = activation_class()
        
        self.conv2 = nn.Conv2d(planes, planes, kernel_size=3, stride=1, padding=1, bias=False)
        self.bn2 = nn.BatchNorm2d(planes)
        self.act2 = activation_class()

        self.shortcut = nn.Sequential()
        if stride != 1 or in_planes != self.expansion * planes:
            self.shortcut = nn.Sequential(
                nn.Conv2d(in_planes, self.expansion * planes, kernel_size=1, stride=stride, bias=False),
                nn.BatchNorm2d(self.expansion * planes)
            )

    def forward(self, x):
        out = self.act1(self.bn1(self.conv1(x)))
        out = self.bn2(self.conv2(out))
        out += self.shortcut(x)
        out = self.act2(out)
        return out

class ResNet(nn.Module):
    def __init__(self, block, num_blocks, num_classes=10, activation_class=nn.ReLU):
        super(ResNet, self).__init__()
        self.in_planes = 64
        self.activation_class = activation_class
        self.conv1 = nn.Conv2d(3, 64, kernel_size=3, stride=1, padding=1, bias=False)
        self.bn1 = nn.BatchNorm2d(64)
        self.act1 = activation_class()
        
        self.layer1 = self._make_layer(block, 64, num_blocks[0], stride=1)
        self.layer2 = self._make_layer(block, 128, num_blocks[1], stride=2)
        self.layer3 = self._make_layer(block, 256, num_blocks[2], stride=2)
        self.layer4 = self._make_layer(block, 512, num_blocks[3], stride=2)
        
        self.avgpool = nn.AdaptiveAvgPool2d((1, 1))
        self.linear = nn.Linear(512 * block.expansion, num_classes)

    def _make_layer(self, block, planes, num_blocks, stride):
        strides = [stride] + [1]*(num_blocks-1)
        layers = []
        for stride in strides:
            layers.append(block(self.in_planes, planes, stride, self.activation_class))
            self.in_planes = planes * block.expansion
        return nn.Sequential(*layers)

    def forward(self, x):
        out = self.act1(self.bn1(self.conv1(x)))
        out = self.layer1(out)
        out = self.layer2(out)
        out = self.layer3(out)
        out = self.layer4(out)
        out = self.avgpool(out)
        out = out.view(out.size(0), -1)
        out = self.linear(out)
        return out

def ResNet18(activation_class):
    return ResNet(BasicBlock, [2, 2, 2, 2], activation_class=activation_class)

\end{lstlisting}
\subsection{实验配置}
本实验基于 PyTorch 框架实现,使用云端 Nividia 4090 GPU + 4颗cpu核心 进行训练。主要超参数设置如下:
\begin{itemize}
    \item 批大小 (Batch Size):128
    \item 训练轮数 (Epochs):40
    \item 学习率 (Learning Rate):0.01
    \item 优化器 (Optimizer):SGD / Adam / SGD with Momentum
    \item 激活函数 (Activation Function):ReLU / LeakyReLU / Sigmoid / GELU
\end{itemize}

\section{激活函数的对比与分析}
激活函数为神经网络引入了非线性因素。本实验在保持 ResNet-18 结构和优化策略以及超参数不变的情况下,对比了四种激活函数的性能表现。

\subsection{数学原理}

\subsubsection{Sigmoid}
Sigmoid 函数将输入映射到 $(0, 1)$ 区间:
\begin{equation}
    \sigma(x) = \frac{1}{1 + e^{-x}}
\end{equation}
\textbf{缺点分析}:在 $x$ 绝对值较大时,导数趋近于 0,容易导致梯度消失;且输出不是以 0 为中心的。

\subsubsection{ReLU}
ReLU 是目前最常用的激活函数,公式如下:
\begin{equation}
    \text{ReLU}(x) = \max(0, x)
\end{equation}
\textbf{优点}:计算简单,在正区间解决了梯度消失问题,收敛速度快。

\subsubsection{LeakyReLU}
为了解决 ReLU 在 $x<0$ 时神经元“死亡”的问题,LeakyReLU 引入了一个小的负斜率(实验中默认 slope=0.01):
\begin{equation}
    \text{LeakyReLU}(x) = \begin{cases} 
    x & \text{if } x > 0 \\
    \alpha x & \text{if } x \le 0 
    \end{cases}
\end{equation}

\subsubsection{GELU }
GELU 在 Transformer 和 BERT 等模型中被广泛应用,它结合了随机正则化的思想:
\begin{equation}
    \text{GELU}(x) = x \Phi(x) = x \cdot \frac{1}{2} \left[ 1 + \text{erf}\left( \frac{x}{\sqrt{2}} \right) \right]
\end{equation}


\subsection{实验结果对比}
% 请替换为你实际生成的图片路径
\begin{figure}[H]
    \centering
    % 假设你保存了不同激活函数的对比图,可以使用子图或分别展示

    \includegraphics[width=0.8\textwidth]{figures/Sigmoid_SGD.png}
    \caption{Sigmoid 激活函数下的训练曲线}
    
\end{figure}

\begin{figure}[H]
    \centering
    % 假设你保存了不同激活函数的对比图,可以使用子图或分别展示

    \includegraphics[width=0.8\textwidth]{figures/LeakyReLU_SGD.png}
    \caption{LeakyReLU 激活函数下的训练曲线}
    
\end{figure}
\begin{figure}[H]
    \centering
    % 假设你保存了不同激活函数的对比图,可以使用子图或分别展示

    \includegraphics[width=0.8\textwidth]{figures/ReLU_SGD.png}
    \caption{ReLU 激活函数下的训练曲线}
    
\end{figure}
\begin{figure}[H]
    \centering
    % 假设你保存了不同激活函数的对比图,可以使用子图或分别展示

    \includegraphics[width=0.8\textwidth]{figures/GELU_SGD.png}
    \caption{GELU 激活函数下的训练曲线}
    
\end{figure}

\textbf{结果分析}:


\section{优化策略的对比与分析}
优化器决定了网络参数更新的方式。实验在固定学习率 ($LR=0.01$) 的条件下对比了以下三种优化器。

\subsection{数学原理}

\subsubsection{SGD (随机梯度下降)}
最基本的参数更新方法,每次迭代使用一个 Batch 的数据计算梯度:
\begin{equation}
    w_{t+1} = w_t - \eta \nabla L(w_t)
\end{equation}
其中 $\eta$ 为学习率。

\subsubsection{SGD with Momentum}
为了抑制震荡并加速收敛,引入动量项 $v_t$(实验中 $\mu=0.9$):
\begin{align}
    v_{t+1} &= \mu v_t + \nabla L(w_t) \\
    w_{t+1} &= w_t - \eta v_{t+1}
\end{align}

\subsubsection{Adam (Adaptive Moment Estimation)}
Adam 结合了动量法和 RMSProp,自适应地调整每个参数的学习率。它利用了一阶矩估计 $m_t$ 和二阶矩估计 $v_t$:
\begin{align}
    m_t &= \beta_1 m_{t-1} + (1-\beta_1)g_t \\
    v_t &= \beta_2 v_{t-1} + (1-\beta_2)g_t^2 \\
    \hat{m}_t &= \frac{m_t}{1-\beta_1^t}, \quad \hat{v}_t = \frac{v_t}{1-\beta_2^t} \\
    w_{t+1} &= w_t - \frac{\eta}{\sqrt{\hat{v}_t} + \epsilon} \hat{m}_t
\end{align}

\subsection{实验结果对比}
\begin{figure}[H]
    \centering
    \includegraphics[width=0.8\textwidth]{figures/GELU_Adam.png}
    \caption{}
\end{figure}

\begin{figure}[H]
    \centering
    \includegraphics[width=0.8\textwidth]{figures/GELU_SGD_Momentum.png}
    \caption{}
    
\end{figure}
\begin{figure}[H]
    \centering
    \includegraphics[width=0.8\textwidth]{figures/GELU_SGD.png}
    \caption{}
    
\end{figure}
% 插入优化器对比图
\begin{figure}[H]
    \centering
    \includegraphics[width=0.8\textwidth]{figures/GELU_Adam.png}
    \caption{不同优化器下的准确率变化曲线}

\end{figure}

\textbf{结果分析}:


\section{模块设计创新:Ghost Module}
\subsection{改进动机}
在传统的 CNN 中,特征图通常存在大量的冗余(相似的特征图)。$GhostNet: More Features from Cheap Operations$ 指出,这些冗余特征图可以通过廉价的线性变换从少量“本质特征图”生成,从而减少计算量和参数量。
\subsection{改进原理}
\begin{figure}[H]
    \centering
    \includegraphics[width=0.8\textwidth]{figures/pho.png}
    \caption{ResNet-50 特征图相似性示意图}
\end{figure}
如图,论文作者观察到,在训练好的深度神经网络(如 ResNet-50)中,特征图之间存在惊人的相似性,例如很多特征图都可以由一张图经过微小的变换生成。既然这些图高度相关,那么我们就没有必要用独立的卷积核去学习它们,而是可以通过廉价的线性变换生成这些冗余特征图,从而节省计算资源。
\begin{figure}[H]
    \centering
    \includegraphics[width=0.8\textwidth]{figures/MSE.png}
    \caption{特征图拟合实验结果}
\end{figure}
作者做了一个简单的实验:取 ResNet-50 中的一对相似特征图,尝试用一个微小的深度卷积核去拟合它们之间的变换关系。结果发现,重建误差(MSE)极小 。这证明了复杂的卷积操作生成的很多特征,完全可以用极低成本的线性变换来替代,而不会丢失太多信息。
\begin{figure}[H]
    \centering
    \includegraphics[width=0.6\textwidth]{figures/Ghostmodule.png}
    \caption{Ghost Module 原始结构示意图}
\end{figure}
如图是$GhostNet: More Features from Cheap Operations$论文中的 Ghost Module 结构示意图。在Ghost Module 中,首先通过普通卷积核,且核和通道的数量少于原始计划,生成“本质特征图”。然后通过廉价的线性操作(本实验采用卷积$nn.Conv2d$)生成冗余特征图。最后将两部分特征图拼接,得到最终输出。
\subsection{代码实现}
本实验基于$GhostNet: More Features from Cheap Operations$,设计了 \texttt{GhostModule} 替代 ResNet 中的标准卷积层。核心代码如下:

\begin{lstlisting}[caption=Ghost Module 核心实现代码, label=code:ghost]
class GhostModule(nn.Module):
    def __init__(self, inp, oup, kernel_size=1, ratio=2, dw_size=3, stride=1, relu=True):
        super(GhostModule, self).__init__()
        self.oup = oup
        init_channels = math.ceil(oup / ratio)
        new_channels = init_channels * (ratio - 1)      
        self.primary_conv = nn.Sequential(
            nn.Conv2d(inp, init_channels, kernel_size, stride, kernel_size//2, bias=False),
            nn.BatchNorm2d(init_channels),
            nn.ReLU(inplace=True) if relu else nn.Sequential(),
        )
        self.cheap_operation = nn.Sequential(
            nn.Conv2d(init_channels, new_channels, dw_size, 1, dw_size//2, groups=init_channels, bias=False),
            nn.BatchNorm2d(new_channels),
            nn.ReLU(inplace=True) if relu else nn.Sequential(),
        )
    def forward(self, x):
        x1 = self.primary_conv(x)
        x2 = self.cheap_operation(x1)
        out = torch.cat([x1, x2], dim=1)
        return out[:, :self.oup, :, :] 
\end{lstlisting}




\subsection{改进后的网络}
我们将原 ResNet 中的 $3 \times 3$ 卷积替换为 \texttt{GhostModule},设置 \texttt{ratio=2},意味着理论上减少了一半的计算量。

\subsection{性能评估}
将改进后的 ResNet-Ghost 与原始 ResNet-18 进行对比:

\begin{table}[H]
    \centering
    \caption{ResNet-18 与 ResNet-Ghost 大小对比}
    \begin{tabular}{cccc}
        \toprule
        Model & Parameters \\
        \midrule
        ResNet-18  & 11.17M \\
        ResNet-Ghost & 5.7M \\
        \bottomrule
    \end{tabular}
\end{table}

\begin{figure}[H]
    \centering
    \includegraphics[width=0.7\textwidth]{figures/LeakyReLU_SGD_Ghost.png}
    \includegraphics[width=0.7\textwidth]{figures/LeakyReLU_SGD.png}
    \caption{训练过程对比1:上为 ResNet-Ghost,下为原 ResNet-18}
    \label{fig:ghost_result1}
\end{figure}
\begin{figure}[H]
    \centering
    \includegraphics[width=0.7\textwidth]{figures/GELU_SGD_Momentum_Ghost.png}
    \includegraphics[width=0.7\textwidth]{figures/GELU_SGD_Momentum.png}
    \caption{训练过程对比2:上为 ResNet-Ghost,下为原 ResNet-18}
    \label{fig:ghost_result2}
\end{figure}
\begin{figure}[H]
    \centering
    \includegraphics[width=0.7\textwidth]{figures/Sigmoid_SGD_Ghost.jpg}
    \includegraphics[width=0.7\textwidth]{figures/Sigmoid_SGD.png}
    \caption{训练过程对比3:上为 ResNet-Ghost,下为原 ResNet-18}
    \label{fig:ghost_result3}
\end{figure}

\paragraph{分析}
\begin{itemize}
    \item Ghost 版本在参数量砍半的情况下,最终的测试集准确率几乎没有明显的损失,验证了特征图冗余性的假设。
    \item 虽然参数减少,但是由于Ghost Module 将一个大卷积拆成了两步:少量标准卷积 + Depthwise 卷积 + Concat。这增加了操作的碎片化程度。GPU 需要启动更多的 Kernel,频繁的 Kernel Launch 开销抵消了计算量的减少,这体现在训练时间不降反增上。
    \item 此外,还可以注意到,Ghost 版本的训练曲线更平滑,验证集准确率的波动更小,这可能是因为Ghost Module 自带的“结构正则化”效应,即Ghost Module 限制了模型的表达能力,减少了过拟合的可能性。
\end{itemize}
\section{实验总结}
本次实验通过复现 ResNet-18 并进行模块化改进,得出以下结论:
\begin{enumerate}
    \item 激活函数方面,ReLU 及其变体(LeakyReLU)在深层网络中表现稳定,优于 Sigmoid。
    \item 优化器方面,Adam 收敛速度最快,但在某些情况下 SGD+Momentum 能获得更好的最终泛化能力。
    \item 通过引入 Ghost Module,模型在保持精度的同时,参数利用率得到了提升,验证了特征图冗余性假设的合理性。
\end{enumerate}

\end{document}